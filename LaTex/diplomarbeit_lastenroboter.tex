\documentclass[ngerman,14pt,a4paper]{article}
\usepackage[T1]{fontenc}
\usepackage{graphicx}
\usepackage{fancyhdr}
\usepackage{amssymb}
\usepackage{nameref}
\usepackage{babel}
\usepackage{hyperref}
\usepackage{titlesec}
\usepackage[a4paper, left=3cm, right=3cm, top=3cm, bottom=3cm]{geometry}
\pagestyle{fancy}
\fancyhf{}
\fancyfoot[R]{\thepage}
\titleformat{\section}{\Huge\bfseries}{\thesection}{1em}{}
\title{\textbf{\Huge Diplomarbeit: \\ Lastenroboter}}
\date{}
\begin{document}
	\maketitle
	\begin{center}
		\textbf{Höhere Technische Bundeslehranstalt Graz Gösting}\\
		\textbf{Schuljahr 2024/25}\\[0.5 cm]
		\includegraphics[scale=0.5]{Pictures/bulme_logo}\\[1 cm]
		\begin{tabular}{l l l}
			\textbf{Diplomanden:} & & \textbf{Betreuer:} \\
			Daniel Schauer & 5AHEL & Prof. DI. Gernot Mörtl \\
			Simon Spari & 5AHEL & \\
			Felix Hochegger & 5AHEL & \\
		\end{tabular}
	\end{center}
	\newpage
	\begin{flushleft}
		\textbf{\Huge Eidesstattliche Erklärung}\\[0.5 cm]
	\end{flushleft}
	Wir erklären an Eides statt, dass wir die vorliegende Diplomarbeit selbstständig und ohne fremde Hilfe verfasst, keine anderen als die angegebenen Quellen und Hilfsmittel benutzt und die den benutzten Quellen wörtlich und inhaltlich entnommenen Stellen als solche erkenntlich gemacht haben. 
	\vspace{2cm}
	
	\noindent
	\begin{tabular}{p{7cm} p{7cm}}
		\hrulefill & \hrulefill \\
		Ort, am TT.MM.JJJJ & Daniel Schauer \\
	\end{tabular}
	
	\vspace{2cm}
	
	\noindent
	\begin{tabular}{p{7cm} p{7cm}}
		& \hrulefill \\
		& Simon Spari \\
	\end{tabular}

	\vspace{2cm}
	\noindent
	\begin{tabular}{p{7cm} p{7cm}}
		& \hrulefill \\
		& Felix Hochegger \\
	\end{tabular}
	\newpage
	\textbf{\Huge Danksagung}\\[0.5 cm]
	An dieser Stelle möchten wir unseren aufrichtigen Dank aussprechen.\\[0.5 cm]
	Ein besonderer Dank gilt Herrn Prof. DI Gernot Mörtl für seine wertvolle Unterstützung, seine fachliche Begleitung und seine konstruktiven Anregungen während der gesamten Arbeit. Seine Expertise und sein Engagement haben maßgeblich zum Gelingen dieser Diplomarbeit beigetragen.\\[0.5 cm]
	Ebenso danken wir unseren Freunden, insbesondere Michael Johannes Anderhuber, für seine Unterstützung beim Schweißen des Gehäuses. Sein handwerkliches Geschick und seine Hilfe waren für die Umsetzung unseres Projekts von großem Wert.\\[0.5 cm]
	Unser großer Dank gilt zudem unserem großzügigen Sponsor,"Vogl Baumarkt Rosental", für das Sponsoring des Metalls für das Gehäuse. Durch diese Unterstützung konnten wir unser Projekt in dieser Form verwirklichen.
	\newpage
	\tableofcontents
	\newpage
	\section{Einleitung}
	\subsection{Kurzzusammenfassung}
	In dieser Diplomarbeit wird ein Lastenroboter entwickelt, der bis zu 25 Kilogramm transportieren kann. Der Roboter wird über eine Website gesteuert, die als Steuerungsplattform dient. Zusätzlich ist eine Kamera eingebaut, die den Transportbereich zeigt, sowie eine Waage, die das Gewicht der transportierten Last misst.\\[0.5 cm]
	Ein Schwerpunkt der Arbeit lieg auf der mechanischen Konstruktion des Roboters, bei der ein stabiles Gehäuse aus Stahl gebaut wird, um Sicherheit und Stabilität zu gewährleisten. Außerdem wird eine eigene Platine entwickelt, die die verschiedenen Hardware-Komponenten, wie die Sensoren und die Motoren, steuert und miteinander verbindet.\\[0.5 cm]
	Die Steuerung des Roboters erfolgt über eine Website, die es dem Benutzer ermöglicht, den Roboter zu bedienen und wichtige Daten wie Akkustand und Gewicht abzurufen. Ein besonderer Fokus liegt auch dabei auf der Übertragung des Kamerabildes auf die Web-Oberfläche sowie der Integration einer schwenkbaren Kamera, um eine flexible Sicht auf den Transportbereich zu gewährleisten. Der ESP32 sorgt dafür, dass die Befehle des Benutzers an den Roboter übermittelt werden.\\[0.5 cm]
	Zusätzlich wird eine OnBoard-Software entwickelt, die es ermöglicht, die Sensoren auszulesen und die Motoren als auch die Kamera anzusteuern.\\[0.5 cm]
	\subsection{Abstract}
	This thesis develops a load robot that can carry up to 25 kilograms. The robot is controlled via a website, which serves as the control platform. Additionally, a camera is integrated to display the transport area, as well as a scale to measure the weight of the carried load.\\[0.5 cm]
	A key focus of the work is on the mechanical design of the robot, where a sturdy steel housing is built to ensure safety and stability. Furthermore, a custom circuit board is developed to control and connect the various hardware components, such as sensors and motors.\\[0.5 cm]
	The robot is controlled via a website, which allows the user to operate the robot and access important data such as battery level and weight. A particular focus is also placed on transferring the camera feed to the web interface and integrating a swivel camera to ensure flexible viewing of the transport area. The ESP32 ensures that the user's commands are transmitted to the robot.\\[0.5 cm]
	Additionally, onboard software is developed to read the sensors and control the motors and camera.
	%Namen links unten als muster für später
	\fancyfoot[L]{Daniel Schauer - nur als Muster}
	\thispagestyle{fancy}
	\newpage
	\section{Projektmanagement}
	
		\subsection{Projektteam} %Daniel
		
		\subsection{Projektstrukturplan} %Daniel
		
		\subsection{Meilensteine} %Daniel
		
		\subsection{Kostenaufstellung} %Daniel
	
	\section{Antrieb}
	
		\subsection{Motoren} %Felix
		
		\subsection{Motorentreiber} %Felix
		
		\subsection{Schaltungsaufbau} %Felix
		
		\subsection{Code} %Daniel
	
	\section{Webserver} %Spari
	 
		\subsection{Grundlegende Ziele}
		 
		\subsection{Ideen und Entwürfe}
		
		\subsection{Webserver}
		
			\subsubsection{Webserver Setup}
			
			\subsubsection{SPIFFS Setup}	
			
		\subsection{WebSocket Kommunikation}
		
			\subsubsection{Kommunikation Setup}	
			
			\subsubsection{Message Handling}
			
		\subsection{Kamera}
		
			\subsubsection{Kamera Setup}
			
			\subsection{Videoübertragung}
			
		\subsection{Website}
		
			\subsubsection{Implementierung der Steuerung}
			
			\subsubsection{Echtzeit-Videoanzeige}
			
			\subsubsection{Anzeige von Sensordaten und Systemstatus}			
			
		\subsection{Herausforderungen und Optimierungen}
		
			\subsubsection{Probleme bei der WebSocket Kommunikation}
			
			\subsubsection{Latenz- und Performance Optimierungen}
			
			\subsubsection{(Speicher- und Rechenleistungseinschränkungen des ESP32)}
		\newpage
					
		\subsection{(Fazit und Ausblick)}
		
			\subsubsection{(Mögliche Erweiterungen und Verbesserungen)}
	
	\section{Gehäuse}
	
		\subsection{Planung und Design} %Felix
		
		\subsection{Realisierung} %Felix
	
		\subsection{Materialliste} %Felix
	
	\section{Platine}
	
		\subsection{Grundschaltung} %Felix
		
		\subsection{Circuit Board} %Felix
		
		\subsection{Fertiger Prototyp} %Felix
	
	
	\section{Kamera}
	
		\subsection{Kamera im Überblick} %Daniel
		
		\subsection{Videoübertragung} %Simon
		
		\subsection{Kameraschwenkung} %Felix
		
			\subsubsection{Gehäuse} %Felix
			
			\subsubsection{Servomotor} %Daniel
	
		\subsection{Code} %Daniel und %Simon
		
	
	\section{Sensoren}
	
		\subsection{Abstandsensor} %Daniel
		
		\subsection{Gewichtsmessung}
		
			\subsubsection{Grundprinzip} %Daniel
		
			\subsubsection{Schaltungsaufbau} %Daniel
		
			\subsubsection{Code} %Daniel
	
	
	\section{Entwicklungstools}
	
	 \subsection{Autodesk Fushion}
	 
	 \subsection{Eagle}
	 
\newpage
	 
	 \subsection{VS-Code}
	 	
	 	\subsubsection{Setup}
	 	
	 	\subsubsection{Bibliotheken}
	 
		\subsubsection{verwendete Bibliotheken}
				
	 \subsection{LaTex}
	 
	 \subsection{GitHub}
	
	\section{Abbildungsverzeichnis}
	
	\section{Literaturverzeichnis}
	 
\end{document}